\documentclass[a4paper,12pt]{article}

\usepackage{préambule}

\title{Activité : le compte est bon}
\date{}

\begin{document}

\maketitle

\begin{greybox}
	Fait tes recherches sur le cahier d'exercices avant de noter sur la feuille !
\end{greybox}

\begin{exercice}
	Remplir les trous avec $+, -$ ou $×$ pour que les égalités soient justes :
	% 6 - 5 + 2 × 3
	% 5 + 5 × 6 - 8 + 3
	$$  6 \hspace{2em} 5 \hspace{2em} 2 \hspace{2em} 3 = 7
		\hspace{4em} 5 \hspace{2em} 5 \hspace{2em} 6 \hspace{2em} 8 \hspace{2em} 3 = \the\numexpr 5 + 5 * 6 - 8 + 3 \relax $$
\end{exercice}

\begin{exercice}
	Remplir les trous avec $+, -$ ou $×$ pour que l'égalité soit juste :
	% 1 × 1 + 1 × 1 + 1 × 1
	$$ 1 \hspace{2em} 1 \hspace{2em} 1 \hspace{2em} 1 \hspace{2em} 1 \hspace{2em} 1 = 3 $$

	Peut-on obtenir un nombre plus grand que 6 ? Pourquoi ?
	% Non, car :
	%   - Les multiplications se font en premier, et ne laissent que des '1'. Elles ne font donc que 'perdre' un nombre.
	%   - Pour avoir le nombre le plus grand possible, il faut donc que des additions, et on obtient 6.
\end{exercice}

\centering\squared{\textbf{Note le cours : calcul avec parenthèses}}

\vspace{2em}

\begin{exercice}
	En utilisant les parenthèses et les symboles $+, -$ ou $×$, fait en sorte que l'égalité soit juste :
	% (1 + 1) × (1 + 1) × (1 + 1)
	$$ 1 \hspace{2em} 1 \hspace{2em} 1 \hspace{2em} 1 \hspace{2em} 1 \hspace{2em} 1 = 8 $$
\end{exercice}

\begin{exercice}
	En utilisant les parenthèses et les symboles $+, -, ×$ ou $÷$, fait en sorte que les égalités soient justes :
	% (4 + 1) × 3 = 15
	% 5 ÷ (1 + 1) + 3 × 4 = 14,5
	% (1 + 2 × 5 - 9) × 8 = 16
	% ((1 + 2) × 5 - 9) × 8 = 48
	\begin{align*}
		4 \hspace{2em} 1 \hspace{2em} 3 = 15 \hspace{4em}                              & 5 \hspace{2em} 1 \hspace{2em} 1 \hspace{2em} 3 \hspace{2em} 4 = 14,5 \\ & \\
		1 \hspace{2em} 2 \hspace{2em} 5 \hspace{2em} 9 \hspace{2em} 8 = 16 \hspace{4em} & 1 \hspace{2em} 2 \hspace{2em} 5 \hspace{2em} 9 \hspace{2em} 8 = 48
	\end{align*}
\end{exercice}

\end{document}