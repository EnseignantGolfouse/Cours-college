\documentclass[a4paper,11pt]{article}

\usepackage{préambule}
\graphicspath{{./Images/}} 

\newgeometry{top=1cm,bottom=1cm,left=1cm,right=1cm}

\makeatletter
\renewcommand{\maketitle}{%
{\scriptsize colle dans ton cahier d'exercices}
	\begin{center}
		\LARGE
		\uline{\@title}
		\vspace{0.5em}
	\end{center}
}
\makeatother

\title{Utilisation du tableur}
\author{}
\date{}

\begin{document}

\maketitle

\begin{center}
	\includegraphics[width=0.8\textwidth]{[gray] Introduction tableur - tableur windows.png}
\end{center}

\begin{exercice}\
	\begin{enumerate}
		\item Fait, sans calculatrice, les calculs suivants:
		      \begin{itemize}
			      \item $50 × 31 = $ ...... % 1550
			      \item $2{,}3 × 7{,}9 = $ ........ % 18,17
			      \item $8{,}16 + 9{,}87 = $ ........ % 18,03
		      \end{itemize}
		\item Connecte-toi sur \url{laclasse.com}, va dans \faicon{briefcase} \textbf{docs}, et \squared{\faicon{download} télécharge} le fichier \textbf{Mathématiques → Séance tableur 1 → fiche-de-calcul}. Ouvre ensuite ce fichier, et observe ce qu'il y a dedans. \vspace{1em}

		      Utilise le tableur pour vérifier tes résultats.
		\item En utilisant le tableur, effectue les calculs suivants, et note les résultats :
		      \begin{itemize}
			      \item $(346 × 78) + (346 + 78) = $ ........
			      \item $5{,}8 + (5{,}8 × 45{,}7) = $ ........
			      \item $(57{,}3 × 5{,}6) + (57{,}3 + 5{,}6) = $ ........
			      \item $(78{,}8 × 89{,}7) + 78{,}8 = $ ........
		      \end{itemize}
		\item Clique sur la cellule \squared{C2}, et regarde la zone de \textbf{Formules}. Il y a normalement écrit \squared{=A2*B2}.

		      Remplace ce texte par \squared{=2*B2}, puis appuie sur \textit{Entrée}. Qu'observes-tu ?

		      En utilisant le tableur, effectue les calculs suivants :
		      \begin{itemize}
			      \item $67 × (56 + 67) = $ ......

			      \item $9{,}8 × (34 + 9{,}8) = $ ......

			      \item $596 × (5{,}9 + 596) = $ ......
		      \end{itemize}
	\end{enumerate}
\end{exercice}

\begin{exercice}\
	\begin{enumerate}
		\item Remplis la colonne \textbf{A} avec les nombres suivants : $6$ ; $8$ ; $9{,}6$ ; $12{,}8$ et $14{,}36$ (cellules A2 à A6).
		\item \begin{itemize}
			      \item Sélectionne la cellule A7.
			      \item Dans la zone de formules, clique sur le symbole “$Σ$”, juste à gauche du signe “=”.
			      \item Sélectionne “Somme”, et appuie sur \textit{Entrée}.
		      \end{itemize}
		      Quel est le nombre dans la cellule A7 ? ........
		\item Calcule $6 + 8 + 9{,}6 + 12{,}8 + 14{,}36$ : ......
		\item Remplis la colonne \textbf{B} avec les nombres suivants : $2{,}1$ ; $0{,}8$ ; $10$ ; $3$ et $1{,}6$ (cellules B2 à B6).
		\item Sélectionne la cellule $C2$. Passes ta souris sur le bord inférieur droit de cette cellule : vois-tu quelque chose de spécial ?

		      \textbf{Clique sur ta souris et maintiens}, puis bouge ta souris vers le bas. Qu'observes-tu ?
		\item Sans changer d'autres choses dans le tableur, trouve le résultat de $14{,}36 × 1{,}6$ : ........
	\end{enumerate}
\end{exercice}

\end{document}