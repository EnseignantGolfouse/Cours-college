\documentclass[12pt]{beamer}

\usepackage{préambule}

\title{Activité : Les deux nombres}
\author{}
\date{}

\begin{document}

\begin{frame}
	\maketitle
\end{frame}

\begin{frame}
	Trouver deux nombres dont la somme est égale à 100.
\end{frame}

\begin{frame}
	On sait que $a + b = 100$. \textbf{Sans remplacer $a$ et $b$ par un nombre particulier}, trouver combien vaut chacune des expressions suivantes :

	\renewcommand{\tabcolsep}{1.2em}
	\begin{tabular}{ll}
		$b + a$                           & $a + b + 50$        \\
		$25 + b + 75 + a$                 & $7{,}4 + a + b + 2{,}6$ \\
		\phantom{$a + b + b + a + a + b$} &
	\end{tabular}
	\pause
	\begin{tabular}{ll}
		$a + a + b + b$         & $(a + b) × 10$      \\
		$a + b + b + a + a + b$ & $a + b + b + a + a$
	\end{tabular}
\end{frame}

\begin{frame}
	On a deux nombres $c$ et $d$ dont la somme fait $225$. Écrire les nombres suivants en utilisant $c$ et $d$, et éventuellement d'autres nombres : \vspace{1em}

	\renewcommand{\arraystretch}{1.5}
	\begin{tabular}{ll}
		$450$    & $=$ \\
		$230$    & $=$ \\
		$100$    & $=$ \\
		$45$     & $=$ \\
		$225000$ & $=$
	\end{tabular}
\end{frame}

\end{document}