\documentclass[a4paper,12pt]{article}

\usepackage{préambule}
\usepackage{clipboard}

\setmathfont[range=it]{Tex Gyre Schola Math}

\TitreDActivite{Activité : Abonnement}

\newcommand{\correction}[1]{{\color{red}#1}}

\begin{document}

\maketitle

\begin{enonce}
	Ivan et Kevin vont souvent regarder des films ensemble.

	\begin{itemize}
		\item Ivan paie à chaque fois sa place à plein tarif : 7€ la place.
		\item Kevin, lui, a un abonnement : il l'a acheté 20€, mais il paie chaque place 4,50€.
	\end{itemize}
\end{enonce}

\begin{enumerate}
	\item Si ils vont voir 3 films :
	      \begin{enumerate}
		      \item Combien paie Ivan ? \correction{21€}
		      \item Combien paie Kevin ? \correction{33,50€}
		      \item Lequel paie le moins ? \correction{Ivan}
	      \end{enumerate}
	\item Si ils vont voir 10 films :
	      \begin{enumerate}
		      \item Combien paie Ivan ? \correction{70€}
		      \item Combien paie Kevin ? \correction{65€}
		      \item Lequel paie le moins ? \correction{Kevin}
	      \end{enumerate}
	\item On appelle $n$ le nombre de films qu'ils vont voir.
	      \begin{enumerate}
		      \item Écrire une expression pour le prix que paie Ivan. \correction{n × 7€}
		      \item Écrire une expression pour le prix que paie Kevin. \correction{20€ + n × 4,50€}
	      \end{enumerate}
	\item Remplir le tableau suivant :

	      \begin{center}
		      \renewcommand{\arraystretch}{1.5}
		      \begin{tabular}{|c|c|c|}
			      \hline
			      Nombre de places & Prix pour Ivan   & Prix pour Kevin
			      \\ \hline
			      3                & \correction{21€} & \correction{33,50€}
			      \\ \hline
			      4                & \correction{28€} & \correction{38€}
			      \\ \hline
			      5                & \correction{35€} & \correction{42,50€}
			      \\ \hline
			      6                & \correction{42€} & \correction{47€}
			      \\ \hline
			      7                & \correction{49€} & \correction{51,50€}
			      \\ \hline
			      8                & \correction{56€} & \correction{56€}
			      \\ \hline
			      9                & \correction{63€} & \correction{60,50€}
			      \\ \hline
			      10               & \correction{70€} & \correction{65€}
			      \\ \hline
		      \end{tabular}
	      \end{center}

	      Y-a-t'il un nombre de places, pour lequel les deux paie le même prix ? \correction{8}
	\item Les expressions $7n$ et $20 + 4{,}5n$ sont elles toujours égales ? \correction{Non, elles ne sont pas égales pour $n = 3$ par exemple.}
	\item Les expressions $7n$ et $20 + 4{,}5n$ sont elles égales pour un $n$ particulier ? \correction{Oui, elles sont égales pour $n = 8$.}
\end{enumerate}

\end{document}