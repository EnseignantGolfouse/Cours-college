\documentclass[a4paper,11pt]{beamer}

\usepackage{préambule}

\begin{document}

\begin{frame}
	\begin{cours}[écriture fractionnaire]
		Soient a et b deux nombres, avec b non égal à 0. Le quotient de a par b est le nombre qui, multiplié par b, donne a.

		On peut le noter :
		\begin{itemize}
			\item a $÷$ b : c'est l'écriture  .............................. % décimale
			      \vspace{0.5em}
			\item $\cfrac{\text{a}}{\text{b}}$ : c'est l'écriture .............................. % fractionnaire

			      a est le .............................. % numérateur

			      b est le .............................. % dénominateur
		\end{itemize}
	\end{cours}

	\begin{exemple}
		Le quotient de 8 par 9 est ......, et on a $..... × ..... = .....$.
		% \frac{8}{9} / \frac{8}{9} × 9 = 8
	\end{exemple}
\end{frame}

\begin{frame}
	\begin{cours}[Fractions]
		Lorsque a et b sont des nombres \textit{entiers}, on dit que $\cfrac{\text{a}}{\text{b}}$ est une .............................. % fraction
	\end{cours}
\end{frame}

\end{document}