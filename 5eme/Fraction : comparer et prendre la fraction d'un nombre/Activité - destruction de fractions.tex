\documentclass[a4paper,11pt,landscape,twocolumn]{article}

\usepackage{préambule}
\usepackage{clipboard}

\setlength{\columnsep}{0.6cm}

\TitreDActivite{Activité : Destruction de fraction}

\begin{document}

\Copy{Activité}{
	\maketitle

	\begin{enonce}
		On va faire un jeu qui se joue à deux :

		On dispose d'une fraction, ainsi que de plusieurs nombres.

		Chacun son tour, un joueur peut:
		\begin{enumerate}
			\item Simplifier la fraction si il le souhaite, puis
			\item Utiliser un des nombres pour l'ajouter ou l'enlever au numérateur ou au dénominateur.
		\end{enumerate}
	\end{enonce}

	Prenez une feuille pour noter le déroulement de la partie !

	\vspace{1em}

	\textbf{\uline{Parties :}}
	\begin{enumerate}
		\item La fraction est $\dfrac{10}{15}$, les nombres sont 3, 3, 5 et 7. \vspace{0.5em}
		\item La fraction est $\dfrac{32}{16}$, les nombres sont 1, 1, 3, 5, 7 et 9. \vspace{0.5em}
		\item La fraction est $\dfrac{50}{32}$, les nombres sont 1, 2, 5, 6, 7, 8, 10. \vspace{0.5em}
		\item La fraction est $\dfrac{117}{57}$, les nombres sont 1, 2, 3, 3, 7, 11, 13, 14, 39. \vspace{0.5em}
		\item La fraction est $\dfrac{258}{175}$, les nombres sont 1, 2, 3, 3, 5, 5, 7, 10, 11, 12, 13. \vspace{0.5em}
		\item La fraction est $\dfrac{2 550}{7 872}$, les nombres sont 1, 1, 2, 2, 2, 7, 8, 9, 10, 12, 15, 20, 21.
	\end{enumerate}
}

\newpage

\Paste{Activité}


\end{document}