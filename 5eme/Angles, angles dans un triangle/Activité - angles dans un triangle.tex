\documentclass[a4paper,11pt]{article}

\usepackage{préambule}
\usetikzlibrary{angles,arrows,arrows.meta,calc,intersections,quotes}

\makeatletter
\renewcommand{\maketitle}{%
{\scriptsize colle dans ton cahier d'exercices}
	\begin{center}
		\LARGE
		\uline{\@title}
		\vspace{0.5em}
	\end{center}
}
\makeatother

\title{Activité : angles dans un triangle}
\date{}
\author{}

\begin{document}

\maketitle

\section{Trouver une propriété}

À l'aide d'un rapporteur, remplis le tableau pour les trois triangles ci-dessous :

\renewcommand{\arraystretch}{1.5}
\begin{center}
	\begin{tabular}{|l|c|c|c|c|}
		\hline
		           & $\widehat{BAC}$ & $\widehat{CBA}$ & $\widehat{ACB}$ & Somme des trois angles \\ \hline
		Triangle 1 & \phantom{30°}   & \phantom{60°}   & \phantom{90°}   & \phantom{180°}         \\ \hline
		Triangle 2 & \phantom{80°}   & \phantom{50°}   & \phantom{50°}   & \phantom{180°}         \\ \hline
		Triangle 3 & \phantom{100°}  & \phantom{20°}   & \phantom{60°}   & \phantom{180°}         \\ \hline
	\end{tabular}
\end{center}

\begin{center}
	\begin{tikzpicture}[scale=0.8]
		\coordinate (A1) at (7,0);
		\coordinate (B1) at (3,3.5);
		\coordinate (C1) at (2.47,0.93);

		\coordinate (A2) at (11,0);
		\coordinate (B2) at (13,4);
		\coordinate (C2) at (15.29,-1.28);

		\coordinate (A3) at (9,-3.5);
		\coordinate (B3) at (3,-3.5);
		\coordinate (C3) at (9.41,-1.17);

		\draw (A1) -- (B1) -- (C1) -- cycle;
		\draw (A2) -- (B2) -- (C2) -- cycle;
		\draw (A3) -- (B3) -- (C3) -- cycle;
		\node at (3.9, 1.3) {Triangle 1};
		\node at (13, 0.7) {Triangle 2};
		\node at (7, -3) {Triangle 3};

		\node[below] at (A1) {A};
		\node[left] at (B1) {B};
		\node[left] at (C1) {C};

		\node[below] at (A2) {A};
		\node[left] at (B2) {B};
		\node[right] at (C2) {C};

		\node[right] at (A3) {A};
		\node[left] at (B3) {B};
		\node[above] at (C3) {C};
	\end{tikzpicture}
\end{center}

\vspace{1em}
Quelle remarque peux-tu faire ? \dotfill

\section{Prouver la propriété}

\squared{Récupère une feuille de dessin auprès du professeur.}

\vspace{0.5em}
\uline{\large \textbf{Construction}}

\begin{enumerate}
	\item Sur la feuille de dessin, trace un triangle de ton choix (à la règle). Appelle ses sommets 'A', 'B' et 'C'.
	\item Avec une équerre, trace une droite \textbf{perpendiculaire à [BC]} et \textbf{passant par A}. Note \textbf{H} le point d'intersection avec [BC].
	\item Découpe ton triangle. Colorie l'angle de A en {\color{red}rouge}, l'angle de B en {\color{blue}bleu} et l'angle de C en {\color{green}vert}.
	\item Colorie de même les angles sur \textbf{l'autre face} du triangle.
	\item Plie le triangle pour ramener le point A sur le point H.
	\item Plie le triangle pour ramener le point B sur le point H.
	\item Plie le triangle pour ramener le point C sur le point H.
	\item \squared{Colle le triangle dans ton cahier d'exercices, de manière à pouvoir le plier.}
\end{enumerate}

\uline{\large \textbf{Observations}}

\begin{itemize}
	\item Que forment les angles obtenus en H ? \dotfill
	\item En déduire la formule $\widehat{BAC} + \widehat{CBA} + \widehat{ACB} = $ \dotfill
\end{itemize}


\end{document}