\documentclass[a4paper,11pt]{article}

\usepackage{préambule}
\usetikzlibrary{angles,quotes,intersections}

\mdfdefinestyle{rappelstyle}{
    style=greyboxstyle,
    frametitle={Rappel},
}
\newmdenv[style=rappelstyle]{rappel}

\title{Introduction}
\date{}
\author{}

\begin{document}

\maketitle

\begin{vocabulaire}[(Rappel)]
	\begin{itemize}
		\item Deux droites sont \textbf{perpendiculaires} si elles forment un angle droit.
		\item Deux droites sont \textbf{sécantes} si elles s'intersectent en un point.
		\item Deux droites sont \textbf{parallèles} si elles ne s'intersectent jamais.
	\end{itemize}
\end{vocabulaire}

\begin{center}
	\begin{tikzpicture}
		\draw[Green,name path=d4] (-4,3) node[above right] {(d₄)} -- (4,-3);
		\draw[Green,name path=d8] (-3,-4) node[above left] {(d₈)} -- (3,4);
		\draw[Green] (-0.4,0.3) -- (-0.1,0.7) -- (0.3,0.4);

		\draw[blue,name path=d1] (-5,2) node[above right] {(d₁)} -- (7,0);
		\draw[blue,name path=d3] (-5,-1) node[above right] {(d₃)} -- (7,-3);
		\draw[name path=d6] (0.5,-5) node[left] {(d₆)} -- (2,4);
		\draw[blue] (1,-2) ++(-0.3,0.05) -- ++(-0.05,-0.3) -- ++(0.3,-0.05);
		\draw[blue] (1.48,0.9) ++(-0.3,0.05) -- ++(-0.05,-0.3) -- ++(0.3,-0.05);

		\draw[name path=d2] (-5,0) -- (7,-2.5) node[above left] {(d₂)};

		\draw[red,name path=d7] (-1,4) node[below right] {(d₇)} -- (-2,-5);
		\draw[red,name path=d5] (3.2,4) node[below right] {(d₅)} -- (2,-5);
	\end{tikzpicture}
\end{center}

Sur la figure ci-dessus :
\begin{itemize}
	\item (d₁) et (d₆) sont \textbf{perpendiculaires}.
	\item (d₁) et (d₃) sont \textbf{parallèles}.
	\item (d₁) et (d₂) ne sont \textbf{ni perpendiculaires, ni parallèles}.
	\item (d₅) et (d₇) ne sont \textbf{ni perpendiculaires, ni parallèles}. On peut avoir l'impression qu'elles sont parallèles, mais il n'y a pas de notation sur la figure nous l'assurant.
	\item (d₆) et (d₇) sont ne sont \textbf{ni perpendiculaires, ni parallèles}. On peut avoir l'impression qu'elles sont parallèles, mais il n'y a pas de notation sur la figure nous l'assurant.
	\item (d₄) et (d₈) sont \textbf{perpendiculaires}.
	\item (d₃) et (d₆) sont \textbf{perpendiculaires}.
\end{itemize}

\begin{greybox}
	Deux droites sont parallèles si une \textbf{troisième droite} est perpendiculaire à chacune de ces deux droites.
\end{greybox}

\section*{dézoom de la figure}

\begin{center}
	\begin{tikzpicture}[scale=0.12]
		\draw[Green,name path=d4] (-4,3) -- (4,-3);
		\draw[Green,name path=d8] (-3,-4) -- (3,4);
		\draw[Green] (-0.4,0.3) -- (-0.1,0.7) -- (0.3,0.4);

		\draw[blue,name path=d1] (-5,2) -- (7,0);
		\draw[blue,name path=d3] (-5,-1) -- (7,-3);
		% (1, 6)
		\draw[name path=d6] (-34,-212) -- (2,4);
		\draw[blue] (1,-2) ++(-0.3,0.05) -- ++(-0.05,-0.3) -- ++(0.3,-0.05);
		\draw[blue] (1.48,0.9) ++(-0.3,0.05) -- ++(-0.05,-0.3) -- ++(0.3,-0.05);

		\draw[name path=d2] (-5,0) -- (7,-2.5);

		\draw[red,name path=d7] (-1,4) -- (-25,-212);
		\draw[red,name path=d5] (3.2,4) -- (-25.6,-212);
	\end{tikzpicture}
\end{center}



\end{document}