\documentclass[a4paper,12pt]{article}

\usepackage{préambule}
\usepackage{préambule-figures}
\usetikzlibrary{calc,positioning}

\title{Patron d'un pavé droit}
\author{}
\date{}

\begin{document}

\maketitle

On fait le patron du pavé droit suivant :

\begin{center}
	\newcommand{\Longeur}{4cm}
	\newcommand{\Largeur}{2cm}
	\newcommand{\Profondeur}{5cm}
	\begin{tikzpicture}
		\paveDroit{\Largeur}{\Longeur}{(1,1)}

		\draw[<->] (0,-0.2) -- node[midway,below] {$\Longeur$} ++(\Longeur,0);
		\draw[<->] (-0.2,0) -- node[midway,left] {$\Largeur$} ++(0,\Largeur);
		\draw[<->] (0.15cm + \Longeur,-0.15) -- node[midway,below right] {$\Profondeur$} ++(1,1);
	\end{tikzpicture}
	\vspace{1em}

	\begin{tikzpicture}
		\draw[\myArrow] (0,0) -- (0,-2);
	\end{tikzpicture}
	\vspace{1.5em}

	\begin{tikzpicture}
		\patronPaveDroit{\Largeur}{\Longeur}{\Profondeur}
		
		\draw[<->] (-0.2,0) -- node[midway,left] {$\Profondeur$} ++(0,-\Profondeur);
		\draw[<->] (0,0.2) -- node[midway,above] {$\Longeur$} ++(\Longeur,0);
		\draw[<->] (\Longeur + \Largeur - 0.2cm,0) -- node[midway,left] {$\Largeur$} ++(0,\Largeur);
	\end{tikzpicture}
\end{center}




\end{document}