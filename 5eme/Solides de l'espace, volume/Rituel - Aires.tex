\documentclass[a4paper,10pt]{beamer}

\usepackage{préambule}
\usetikzlibrary{calc}

\title{Aires}
\author{}
\date{}

\begin{document}

\begin{frame}
	Reproduit ces figures dans ton cahier d'exercices, et donne leurs \textbf{aires} :

	\begin{center}
		\newcommand{\FigureAWidth}{1.5}
		\newcommand{\FigureAHeigth}{1}
		\newcommand{\FigureBWidth}{1.575}
		\newcommand{\FigureBHeigth}{2}
		\newcommand{\FigureCWidth}{2}
		\newcommand{\FigureCHeigth}{2}
		\newcommand{\FigureDRadius}{1}

		\begin{tikzpicture}
			\coordinate (FigureA) at (0,0);
			\coordinate (FigureB) at (4.5,0);
			\coordinate (FigureC) at (0,-4);
			\coordinate (FigureD) at (4.5,-4);

			\draw (FigureA) rectangle ++(\FigureAWidth,\FigureAHeigth);
			\draw[<->] (FigureA) ++(-0.2,0) -- node[midway,left] {$3$cm} ++(0,\FigureAHeigth);
			\draw[<->] (FigureA) ++(0,-0.2) -- node[midway,below] {$4,5$cm} ++(\FigureAWidth,0);
			\node at ($(FigureA) + (\FigureAWidth / 2,-1)$) {Aire = .........};

			\draw (FigureB) rectangle ++(\FigureBWidth,\FigureBHeigth);
			\draw[<->] (FigureB) ++(-0.2,0) -- node[midway,left] {$16$cm} ++(0,\FigureBHeigth);
			\draw[<->] (FigureB) ++(0,-0.2) -- node[midway,below] {$12,5$cm} ++(\FigureBWidth,0);
			\node at ($(FigureB) + (\FigureBWidth / 2,-1)$) {Aire = .........};

			\draw (FigureC) -- ++(\FigureCWidth,0) -- ++(-\FigureCWidth / 2,\FigureCHeigth) -- cycle;
			\draw[<->] (FigureC) ++(-0.2,0) -- node[midway,left] {$8$cm} ++(0,\FigureCHeigth);
			\draw[<->] (FigureC) ++(0,-0.2) -- node[midway,below] {$5$cm} ++(\FigureCWidth,0);
			\node at ($(FigureC) + (\FigureCWidth / 2,-1)$) {Aire = .........};

			\draw ($(FigureD) + (\FigureDRadius,\FigureDRadius)$) circle (\FigureDRadius);
			\draw ($(FigureD) + (\FigureDRadius,\FigureDRadius)$) -- ++(0,\FigureDRadius);
			\node at ($(FigureD) + (\FigureDRadius,\FigureDRadius)$) {×};
			\draw[<->] ($(FigureD) + (2*\FigureDRadius,\FigureDRadius)$) ++(0.2,0) -- node[midway,right] {$6$cm} ++(0,\FigureDRadius);
			\node at ($(FigureD) + (\FigureDRadius,-1)$) {Aire = .........};
		\end{tikzpicture}
	\end{center}
\end{frame}

\end{document}