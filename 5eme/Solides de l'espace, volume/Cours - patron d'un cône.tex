\documentclass[a4paper,10pt]{article}

\usepackage{préambule}
\usepackage{clipboard}

\begin{document}

\Copy{Cours}{
\begin{cours}[Patron du cône]
	Le patron d'un cône est :

	\begin{center}
		\newcommand{\racinededeux}{1.4142135623730951}

		\begin{tikzpicture}
			\draw[fill=green!30]
			(0,0) coordinate (S)
			-- (3,0) coordinate (A)
			arc (0:120:3) coordinate (B)
			-- cycle;
			\draw[fill=blue!30] (4 / \racinededeux,4 / \racinededeux) coordinate (H) circle (1);
			\draw[very thick,draw=Purple] (0.7,0) arc(0:120:0.7);
			\node at (0.3,0.9) {\color{Purple}$α$};
			\draw[<->] (0,-0.2) -- node[midway,below] {$g$} ++(3,0);
			\draw[<->] (4 / \racinededeux,4 / \racinededeux + 1.2) -- node[midway,above] {$r$} ++(1,0);
			\draw[very thin,dashed] (4 / \racinededeux,4 / \racinededeux + 1.2) -- ++(0,-1.2) ++(1,0) -- ++(0,1.2);

			\foreach \p/\pos in {S/below left,A/below right,B/left,H/below} {
				\node at (\p) {×};
				\node[\pos] at (\p) {\p};
			}
		\end{tikzpicture}
	\end{center}

	Pour dessiner le patron d'un cône dont :
	\begin{itemize}
		\item La longueur de la génératrice est $g$
		\item le rayon de la base est $r$
	\end{itemize}
	on doit :
	\begin{itemize}
		\item Dessiner une ligne de longueur $g$.
		\item Utiliser un tableau de proportionnalité pour connaitre l'angle $α$ :
		
		\begin{tikzpicture}
			\node (TAB) at (0,0) {\begin{tabular}{|l|c|c|}
				\hline
				Mesure de l'angle en ° : & 360 & $α$ \\ \hline
				Longeur de l'arc : & $2 × π × g$ & $2 × π × r$ \\ \hline
			\end{tabular}};
			\draw[thick,Purple,->] (4.8,-0.3) arc(-90:90:0.3) node[midway,right] {{\large$×$} $\dfrac{360}{2 × π × g}$};
		\end{tikzpicture}
		\item Dessiner un cercle de rayon $r$ et de centre H, sachant que la distance [SH] est égale à $g + r$.
	\end{itemize}
\end{cours}
}

\Paste{Cours}

\end{document}