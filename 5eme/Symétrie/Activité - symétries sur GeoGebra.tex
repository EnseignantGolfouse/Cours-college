\documentclass[a4paper,12pt]{article}

\usepackage{préambule}

\makeatletter
\renewcommand{\maketitle}{%
	\begin{center}
		\LARGE
		\uline{\@title}
		\vspace{1em}
	\end{center}
}
\makeatother

\title{Activité : Symétries sur GeoGebra}
\date{}
\author{}

\begin{document}

\maketitle

\begin{attention}
	Quand tu as créé une figure, essaie de bouger les point pour vérifier que ça ne casse pas.
\end{attention}

\begin{exercice}
	Commence une nouvelle figure dans GeoGebra.
	\begin{enumerate}
		\item Crée une droite $(d)$, formée de deux points $A$ et $B$.
		\item Place un point $C$ sur cette droite, et un point $O$ en dehors.
		\item Crée les points $A'$, $B'$ et $C'$ symétriques de $A$, $B$ et $C$ par rapport à $O$.

		      Crée la droite $(d')$, symétrique de $(d)$ par rapport à $O$.
		\item Trace le segment $[AA']$, et vérifie que $O$ est le milieu de ce segment. \vspace{2em}
		\item Que peux-tu dire de $A'$, $B'$ et $C'$ par rapport à $(d')$ ? \vspace{2em}
		\item Que remarques-tu à propos des droites $(d)$ et $(d')$ ? \vspace{2em}
		\item Que se passe-t-il si le point $O$ est sur la droite $(d)$ ? \vspace{2em}
	\end{enumerate}
\end{exercice}

\begin{exercice}
	\begin{enumerate}
		\item Crée un segment $[AB]$, et un point $O$ n'importe où.
		\item Construit $I$ le milieu de $[AB]$.
		\item Construit $A'$, $B'$ et $I'$ symétriques de $A$, $B$ et $I$ par rapport à $O$.
		\item Que peut-on dire de $I'$ par rapport à $[A'B']$ ? \vspace{2em}
	\end{enumerate}
\end{exercice}

\begin{exercice}
	\begin{enumerate}
		\item Construit quatre points $A$, $B$, $C$ et $O$.
		\item Construit les segments $[AB]$ et $[BC]$.
		\item Construit les points $A'$, $B'$ et $C'$ symétriques de $A$, $B$ et $C$ par rapport à $O$.

		      Que peut-on dire des longueurs $AB$ et $A'B'$ ? \vspace{2em}

		      Que peut-on dire des longueurs $BC$ et $B'C'$ ? \vspace{2em}
	\end{enumerate}
\end{exercice}

\begin{exercice}
	\begin{enumerate}
		\item Construit quatre points $A$, $B$, $C$ et $O$.
		\item Construit les points $A'$, $B'$ et $C'$ symétriques de $A$, $B$ et $C$ par rapport à $O$.
		\item Construit les angles $\widehat{ABC}$ et $\widehat{A'B'C'}$.

		      Que peut-on dire de ces deux angles ?
	\end{enumerate}
\end{exercice}

\end{document}