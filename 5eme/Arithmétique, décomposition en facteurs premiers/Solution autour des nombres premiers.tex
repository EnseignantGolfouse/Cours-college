\documentclass[a4paper,12pt]{article}

\usepackage{préambule}

%%%%%%%%%%%%%%%%%%

\title{Autour des nombres premiers}
\date{}
\author{}

%%%%%%%%%%%%%%%%%%


\begin{document}

\maketitle
%\tableofcontents % pour avoir une table des matières

\begin{exercice}
	\begin{enumerate}
		\item 25
		\item 101 103 107 109 113 127
	\end{enumerate}
\end{exercice}

\begin{exercice}
	3 11 17
\end{exercice}

\begin{exercice}
	\begin{enumerate}
		\item 1, 2, 3, 4, 6, 12
		\item 2, 3
		\item tous les diviseurs (sauf 1) sont des multiple des diviseurs premiers.
	\end{enumerate}
\end{exercice}

\begin{exercice}[(avancé)\\]
	Infini \\
	Car $p1×p2×⋯×pₙ + 1$ serait alors premier.
\end{exercice}

\end{document}