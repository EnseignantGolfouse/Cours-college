\documentclass[a4paper,12pt]{article}

\usepackage{préambule}

%%%%%%%%%%%%%%%%%%

\title{Exercices : Division euclidiennes, nombres premiers}
\date{10 septembre 2021}
\author{}

%%%%%%%%%%%%%%%%%%


\begin{document}

\maketitle

\begin{exercice}
	Faire \textbf{sans} calculatrice les divisions euclidiennes suivantes :
	\begin{enumerate}
		\begin{minipage}{0.45\linewidth}
			\item $\begin{array}{r|r}
					423 & 5 \\
					\cline{2-2}
					    &
				\end{array}$
			\item $\begin{array}{r|r}
					2251 & 10 \\
					\cline{2-2}
					     &
				\end{array}$
			\item $\begin{array}{r|r}
					1174 & 11 \\
					\cline{2-2}
					     &
				\end{array}$
			\item $\begin{array}{r|r}
					3201 & 14 \\
					\cline{2-2}
					     &
				\end{array}$
		\end{minipage}
		\begin{minipage}{0.45\linewidth}
			\item $\begin{array}{r|r}
					745 & 8 \\
					\cline{2-2}
					    &
				\end{array}$
			\item $\begin{array}{r|r}
					8756 & 4 \\
					\cline{2-2}
					     &
				\end{array}$
			\item $\begin{array}{r|r}
					4985 & 23 \\
					\cline{2-2}
					     &
				\end{array}$
			\item $\begin{array}{r|r}
					874 & 17 \\
					\cline{2-2}
					    &
				\end{array}$
		\end{minipage}
	\end{enumerate}
\end{exercice}

\begin{exercice}
	Faire \textbf{avec} la calculatrice les divisions euclidiennes suivantes :
	\begin{enumerate}
		\begin{minipage}{0.45\linewidth}
			\item $\begin{array}{r|r}
					894523 & 452 \\
					\cline{2-2}
					       &
				\end{array}$
			\item $\begin{array}{r|r}
					54779612 & 54236 \\
					\cline{2-2}
					         &
				\end{array}$
		\end{minipage}
		\begin{minipage}{0.45\linewidth}
			\item $\begin{array}{r|r}
					4215637 & 3156 \\
					\cline{2-2}
					        &
				\end{array}$
			\item $\begin{array}{r|r}
					7530633 & 2013 \\
					\cline{2-2}
					        &
				\end{array}$
		\end{minipage}
	\end{enumerate}
\end{exercice}

\begin{exercice}
	Dire si les affirmations suivantes sont vraies ou fausses, et expliquer pourquoi. Ne pas utiliser la calculatrice.
	\begin{enumerate}
		\item $235$ est un multiple de $3$
		\item $712$ est un multiple de $5$
		\item $1022$ est un multiple de $2$
		\item $17$ est un diviseur de $1717$
		\item $11$ est un multiple de $1111$
		\item $5139$ est un multiple de $16$
	\end{enumerate}
\end{exercice}

\begin{exercice}
	\begin{itemize}
		\item Combien y-a-t'il de minutes dans une journée ? Et combien de secondes ?
		\item Alice a été en cours 420 minutes aujourd'hui. Combien d'heures a-t-elle passé en cours ?
		\item Bilal a lui compté les secondes ! Il en a compté $22320$. A-t-il passé un nombre rond d'heures en cours ? Si non, écrire le temps qu'il a passé en cours en heures et en minutes.
	\end{itemize}
\end{exercice}

\begin{exercice}
	\begin{enumerate}
		\item Lister tous les diviseurs de $140$.
		\item Lister tous les diviseurs de $42$.
		\item Lister tous les nombres qui sont diviseur de $140$ \textbf{et} de $42$.
	\end{enumerate}
\end{exercice}

\textbf{Si j'ai fini, je peux : \\}

\begin{itemize}
	\item Faire mes devoirs.
	\item Faire les exercices 54, 55, 56, 57 et 58 page 28.
\end{itemize}

\end{document}