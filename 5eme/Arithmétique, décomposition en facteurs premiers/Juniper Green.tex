\documentclass[landscape]{article}

\usepackage{préambule}

\begin{document}

\begin{Huge}
	\begin{center}
		\begin{TAB}(r,0cm,0cm)[0pt,10cm,10cm]{|c|c|c|c|c|}{|c|c|c|c|}
			1 & 2 & 3 & 4 & 5 \\
			6 & 7 & 8 & 9 & 10 \\
			11 & 12 & 13 & 14 & 15 \\
			16 & 17 & 18 & 19 & 20 \\
		\end{TAB}
	\end{center}
\end{Huge}

\begin{Large}
	\textbf{Règles :}
	\begin{itemize}
		\item À chaque tour, il faut dire un nombre qui n'a pas encore été entouré.
		\item Le nombre doit être un \textbf{multiple} ou \textbf{diviseur} du nombre précédent.
		\item La partie s'arrête lorsqu'on ne peut plus jouer de coup.
	\end{itemize}
\end{Large}

\end{document}