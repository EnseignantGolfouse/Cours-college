\documentclass[a4paper,12pt]{article}

\usepackage{préambule}

%%%%%%%%%%%%%%%%%%

\title{Autour des nombres premiers}
\date{}
\author{}

%%%%%%%%%%%%%%%%%%


\begin{document}

\maketitle

\begin{exercice}
	\begin{enumerate}
		\item Combien y-a-t'il de nombres premiers inférieurs à 100 ?
		\item Trouver 3 nombres premiers plus grands que 100.
	\end{enumerate}
\end{exercice}

\begin{exercice}
	Trouver tous les diviseurs de 561 qui sont aussi des nombres premiers.
\end{exercice}

\begin{exercice}
	\begin{enumerate}
		\item Faire la liste des diviseurs de 12.
		\item Parmi ces diviseurs, déterminer lesquels sont premiers.
		\item Vérifier que tous les diviseurs autres que $1$ sont multiples d'au moins un de ces diviseurs premiers.
	\end{enumerate}
\end{exercice}

\begin{exercice}[(avancé)]
	Combien y-a-t'il de nombres premiers en tout ? Y en-a-t'il un nombre fini ? \\ \\
	Essayez de trouver une justification pour votre idée.
\end{exercice}

\end{document}