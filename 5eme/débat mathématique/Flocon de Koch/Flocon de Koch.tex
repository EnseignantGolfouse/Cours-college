\documentclass[a4paper,10pt]{beamer}

\usepackage{préambule}
\usetikzlibrary{lindenmayersystems}
\usepackage{tcolorbox}

\tikzset{
  Koch curve/.style = {
    l-system={
      rule set={F -> F+F--F+F},
      axiom=F,
      step=1pt,
      angle=60,
      #1
    }
  }
}

\makeatletter
\renewcommand{\maketitle}{%
	\begin{center}
		\LARGE\@title
		\vspace{1em}
	\end{center}
}
\makeatother

\title{Flocon de Koch}
\date{}
\author{}

\begin{document}

\begin{frame}
	\begin{center}
		{\LARGE Flocon de Koch}

		\vspace{1cm}

		\begin{tikzpicture}
			\draw[Koch curve={order=2,step=180pt/3^(2)}] l-system;
		\end{tikzpicture}
	\end{center}
\end{frame}

\begin{frame}
	\hspace{-2em} 1. Premier dessin
	\begin{tcolorbox}
		\begin{itemize}
			\item[a.] Tracer un segment.
			\item[b.] Le découper en trois parties égales.
			\item[c.] Poser un triangle équilatéral sur le segment du milieu, et effacer sa base.
		\end{itemize}
	\end{tcolorbox}
	Si la longueur du segment de base est donnée, quelle est la longueur de la nouvelle figure ?

	\vspace{0.5em}

	\hspace{-2em} 2. On recommence les étapes b. et c. sur chaque petit segment ;

	puis sur chacun des segments de cette nouvelle figure ;

	et ainsi de suite ...
	\begin{itemize}
		\item[$∙$] Quelle est la longueur de la 4\textsuperscript{ème} figure ?
		\item[$∙$] Quelle est la longueur de la 10\textsuperscript{ème} figure ?
		\item[$∙$] Quelle est la longueur de la 103\textsuperscript{ème} figure ?
		\item[$∙$] Quelle est la longueur de la n\textsuperscript{ème} figure ?
	\end{itemize}
\end{frame}


\end{document}