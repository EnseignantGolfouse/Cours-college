\documentclass[a4paper,12pt]{article}

\usepackage{préambule}

\TitreDActivite{Activité : Multiplier des nombres relatifs}

\begin{document}

\maketitle

On veut savoir comment multiplier des nombres relatifs. Eh bien, on peut le trouver nous mêmes !

\section*{Distributivité}

On connait une règle lorsqu'on multiplie des nombres (positifs) entre eux :

Si $a$, $b$ et $c$ sont trois nombres positifs, alors on a
$$ a × (b + c) = .... × .... + .... × .... $$

\section*{Avec des nombres négatifs}

On va maintenant dire ce qu'on veut : on \textit{voudrait} que l'égalité ci-dessus soit aussi vraie pour des nombres relatifs !

On va alors voir où cela nous mène-t-il :

\begin{itemize}
	\item Si \uline{$a = +1$, $b = +1$, $c = -1$}, l'égalité devient :
	      \vspace{7.5em}

	      Donc $(+1) × (-1) = ....$
	\item Si \uline{$a = -1$, $b = +1$, $c = -1$}, l'égalité devient :
	      \vspace{7.5em}

	      Donc $(-1) × (-1) = ....$
\end{itemize}

\section*{Conclusion}

On peut donc en tirer la règle suivante :

\begin{greybox}[frametitle={Règle}]
	Lorsqu'on multiplie deux nombres relatifs, le signe du résultat est :
	\vspace{0.5em}
	\begin{itemize}
		\item \textbf{positif} si le signe des deux nombres est ............
		      \vspace{0.5em}
		\item \textbf{négatif} si le signe des deux nombres est ............
	\end{itemize}
\end{greybox}

\end{document}