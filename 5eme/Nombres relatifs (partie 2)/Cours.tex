% !TEX root = ./Cours.tex
\documentclass[../€Cours-complet/Cours-complet]{subfiles}

\titleorchapter{Nombres relatifs (partie 2)}{11}

\begin{document}

\maketitleCours

\section{Addition}

\begin{greybox}[frametitle={Rappel}]
	La \textbf{distance à zéro} d'un nombre relatif est le nombre \uline{positif} situé après le signe.
\end{greybox}

\begin{exemple}
	La distance à zéro de $+5$ est $5$.

	La distance à zéro de $-2$ est $2$.
\end{exemple}

\begin{cours}[Addition de nombres relatifs, cas 1]
	Pour additionner deux nombres négatifs :
	\begin{itemize}
		\item On fait la somme de leurs distances à zero. C'est-à-dire, on fait la somme sans les signes “$-$”.
		\item On ajoute un signe “$-$” devant le résultat.
	\end{itemize}
\end{cours}

\begin{exemple}
	Pour calculer $-2 + (-7)$ :
	\begin{itemize}
		\item La somme de leurs distances à zéro est $2 + 7 = 9$.
		\item On rajoute un signe “$-$” devant le résultat.
	\end{itemize}
	Donc la somme de $-2$ et $-7$ est $-9$.
\end{exemple}

\begin{cours}[Addition de nombres relatifs, cas 2]
	Si deux nombres relatifs sont de signes \uline{contraires}, alors leur somme a
	\begin{itemize}
		\item Le signe du nombre qui a la plus grande distance à zéro.
		\item Pour distance à zéro, la \textbf{différence} de leurs distances à zéro.
	\end{itemize}
\end{cours}

\begin{exemple}
	Pour calculer $-8 + 6$ :
	\begin{itemize}
		\item Le nombre qui à la plus grande distance à zéro est $-8$, donc le résultat est \textit{négatif}.
		\item La différence de leurs distances à zéro est $8 - 6 = 2$.
	\end{itemize}
	Donc la somme de $-8$ et de $6$ est $-2$.
\end{exemple}

\section{Nombres opposés}

\begin{cours}
	Deux nombres sont \textbf{opposés} si leur somme est égale à zéro.

	De manière équivalente, deux nombres opposés :
	\begin{itemize}
		\item Sont de signes contraires.
		\item Ont la même distance à zéro.
	\end{itemize}
\end{cours}

\begin{exemple}
	\begin{itemize}
		\item $3,2$ et $-3,2$ sont opposés.
		\item L'opposé de $-4,6$ est $+4,6$ (ou seulement $4,6$).
	\end{itemize}
\end{exemple}

\section{Soustraction}

\begin{cours}[Soustraction de nombres relatifs]
	Pour \textbf{soustraire} un nombre relatif, on ajoute son opposé.
\end{cours}

\begin{exemple}
	\begin{minipage}{0.45\textwidth}
		\begin{align*}
			A & = -5 - 2    \\
			  & = -5 + (-2) \\
			  & = -(5 + 2)  \\
			  & = -7
		\end{align*}
	\end{minipage}
	\begin{minipage}{0.45\textwidth}
		\begin{align*}
			B & = 3 - (-8,7) \\
			  & = 3 + 8,7    \\
			  & = 11,7
		\end{align*}
	\end{minipage}
\end{exemple}

\begin{methode}[Simplification d'écriture]
	Pour transformer des additions et soustractions sur les relatifs en opérations sur des nombres positifs, on applique les règles suivantes :

	\begin{center}
		$+$ suivi de $+$ donne $+$

		$-$ suivi de $-$ donne $+$

		$-$ suivi de $+$ donne $-$

		$+$ suivi de $-$ donne $-$
	\end{center}
\end{methode}

\begin{exemple}
	\begin{itemize}
		\item $5 - (-8)$ : Il y a un $-$ suivi d'un $-$, ce qui donne $+$.

		      Donc $5 - (-8) = 5 + 8 = 13$.
		\item $-3 + (-7)$ : Il y a un $+$ suivi d'un $-$, ce qui donne $-$.

		      Donc $-3 + (-7) = -3 - 7 = -10$.
	\end{itemize}
\end{exemple}

\end{document}