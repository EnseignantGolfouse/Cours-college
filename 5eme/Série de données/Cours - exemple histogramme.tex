\documentclass[a4paper,12pt,landscape,twocolumn]{article}

\usepackage{préambule}
\usepackage{clipboard}

\renewcommand{\arraystretch}{1.3}

\begin{document}

\Copy{exemple}{
	\begin{exemple}
		On a mesuré la taille des élèves dans le collège. Comme presque toutes ces tailles sont différentes, on les a regroupé en \textit{classes}, d'\textbf{amplitude} 5cm :


		\begin{center}
			\begin{tabular}{|l|c|c|c|}
				\hline
				Taille (en cm) & 135    & 140    & 145
				\\
				entre          & et 139 & et 144 & et 149
				\\ \hline
				Effectif       & 75     & 161    & 255
				\\ \hline
				               & 150    & 155    & 160
				\\
				               & et 154 & et 159 & et 164
				\\ \hline
				               & 182    & 117    & 143    % total : 913
				\\ \hline
			\end{tabular}
		\end{center}

		\begin{center}
			\begin{tikzpicture}
				\begin{axis}[
						width = 0.8\textwidth,
						xmin=133, xmax=167,
						ymin=0, ymax=280,
						axis lines=center,
						ylabel={Effectif},
						xlabel={\ Taille},
						xlabel style={below right},
						ylabel style={above left},
						area style,
						xtick={135,140,145,150,155,160,165}
					]
					\addplot+[ybar interval,mark=no] plot coordinates { (135,0) };
				\end{axis}
			\end{tikzpicture}
		\end{center}
	\end{exemple}
}

\newpage
\Paste{exemple}

\end{document}