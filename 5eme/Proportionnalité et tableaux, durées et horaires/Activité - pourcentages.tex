\documentclass[a4paper,12pt]{article}

\usepackage{préambule}
\usetikzlibrary{positioning}

\makeatletter
\renewcommand{\maketitle}{%
{\scriptsize colle dans ton cahier d'exercices, et écrit dans ton cahier} \vspace{0.5em}

	\begin{center}
		\LARGE
		\uline{\@title}
	\end{center}
}
\makeatother

\title{Activité: pourcentages}
\date{}
\author{}

\begin{document}

\maketitle

\begin{attention}[frametitle={Exercice 1}]
	On a dans un entrepôt des boites empilées les unes sur les autres :

	\begin{itemize}
		\item la première fait 4 centimètres de côté.
		\item la deuxième, posée sur la première, fait $50\%$ de la taille de la première.
		\item la troisième, posée \textbf{à droite} de la première, fait $125\%$ de la taille de la première.
		\item la quatrième, posée sur la troisième, fait $40\%$ de la taille de la troisième.
	\end{itemize}

	Dessine toutes ces boîtes sur ton cahier d'exercices.
\end{attention}

\begin{attention}[frametitle={Exercice 2}]
	Une peintre veut faire de gigantesque toiles. Pour cela, elle fabrique elle-même ses couleurs, à partir de peinture \uline{rouge}, \uline{verte}, \uline{bleue} et \uline{noire}.

	La peintre a fourni les proportions de chaque couleur, ainsi que la quantité de noir : \vspace{1em}

	\renewcommand{\arraystretch}{1.2}
	\begin{tabular}{|c|c|c|c|c|c|c|}
		\hline
		                & Rouge   & Vert    & Bleu    & Noir    &  & Quantité de noir \\
		                & (en \%) & (en \%) & (en \%) & (en \%) &  & (en litres)      \\ \hline
		Fuschia         & 32      & 8       & 20      & 40      &  & 20               \\ \hline
		Or              & 33      & 22      & 0       & 45      &  & 15               \\ \hline
		Azur            & 4       & 17      & 26      & 53      &  & 26.5             \\ \hline
		Jaune           & 33      & 33      & 0       & 34      &  & 10               \\ \hline
		Argile          & 31      & 31      & 32      & 6       &  & 12               \\ \hline
		Indigo          & 16      & 4       & 32      & 48      &  & 12               \\ \hline
		Gris anthracite & 6       & 6       & 6       & 82      &  & 41               \\ \hline
	\end{tabular}
	\renewcommand{\arraystretch}{1}

	\begin{center}
		(⚠ Ce n'est pas un tableau de proportionnalité, juste un tableau normal !)
	\end{center}
	\vspace{0.5em}

	Pour chacune des couleurs demandées, calcule combien de litres de rouge, de vert, de bleu et de noir sont nécessaires.
\end{attention}

\end{document}