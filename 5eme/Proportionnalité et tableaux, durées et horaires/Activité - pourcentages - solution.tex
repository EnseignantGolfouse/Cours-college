\documentclass[a4paper,10pt]{beamer}

\usepackage{préambule}

\begin{document}

\begin{frame}
	\frametitle{Exercice 1}

	\begin{itemize}
		\item $50\%$ de 4 centimètres fait 2 centimètres.
		\item $125\%$ de 4 centimètres fait 5 centimètres.
		\item $40\%$ de 5 centimètres fait 2 centimètres.
	\end{itemize}
\end{frame}

\begin{frame}
	\frametitle{Exercice 1}

	\begin{tikzpicture}
		\draw (0,0) -- ++(4,0) -- ++(0,4) -- ++(-4,0) -- cycle;
		\draw (1,4) -- ++(2,0) -- ++(0,2) -- ++(-2,0) -- cycle;

		\draw (5,0) -- ++(5,0) -- ++(0,5) -- ++(-5,0) -- cycle;
		\draw (6,5) -- ++(2,0) -- ++(0,2) -- ++(-2,0) -- cycle;
	\end{tikzpicture}
\end{frame}

{\small
\renewcommand{\arraystretch}{1.5}
\begin{frame}
	\frametitle{Exercice 2}

	\hspace*{-0.5cm}\begin{tabular}{|c|c|c|c|c|c|}
		\hline
		                & Rouge       & Vert        & Bleu        & Noir        & Coefficient de   \\
		                & (en litres) & (en litres) & (en litres) & (en litres) & proportionnalité \\ \hline
		Fuschia         &             &             &             & 20          &                  \\ \hline
		Or              &             &             &             & 15          &                  \\ \hline
		Azur            &             &             &             & 26,5        &                  \\ \hline
		Jaune           &             &             &             & 34          &                  \\ \hline
		Argile          &             &             &             & 12          &                  \\ \hline
		Indigo          &             &             &             & 12          &                  \\ \hline
		Gris anthracite &             &             &             & 41          &                  \\ \hline
	\end{tabular}
\end{frame}

\begin{frame}
	\frametitle{Exercice 2}

	\hspace*{-0.5cm}\begin{tabular}{|c|c|c|c|c|c|}
		\hline
		                & Rouge       & Vert        & Bleu        & Noir        & Coefficient de   \\
		                & (en litres) & (en litres) & (en litres) & (en litres) & proportionnalité \\ \hline
		Fuschia         &             &             &             & 20          & 0,5              \\ \hline
		Or              &             &             &             & 15          & 0,33             \\ \hline
		Azur            &             &             &             & 26,5        & 0,5              \\ \hline
		Jaune           &             &             &             & 34          & 0,29             \\ \hline
		Argile          &             &             &             & 12          & 2                \\ \hline
		Indigo          &             &             &             & 12          & 0,25             \\ \hline
		Gris anthracite &             &             &             & 41          & 0,5              \\ \hline
	\end{tabular}
\end{frame}

\begin{frame}
	\frametitle{Exercice 2}

	\hspace*{-0.5cm}\begin{tabular}{|c|c|c|c|c|c|}
		\hline
		                & Rouge       & Vert        & Bleu        & Noir        & Coefficient de   \\
		                & (en litres) & (en litres) & (en litres) & (en litres) & proportionnalité \\ \hline
		Fuschia         & 16          & 4           & 10          & 20          & 0,5              \\ \hline
		Or              & 11          & 7,3         & 0           & 15          & 0,33             \\ \hline
		Azur            & 2           & 8,5         & 13          & 26,5        & 0,5              \\ \hline
		Jaune           & 9,7         & 9,7         & 0           & 10          & 0,29             \\ \hline
		Argile          & 15,5        & 15,5        & 16          & 12          & 2                \\ \hline
		Indigo          & 4           & 1           & 8           & 12          & 0,25             \\ \hline
		Gris anthracite & 3           & 3           & 3           & 41          & 0,5              \\ \hline
	\end{tabular}
\end{frame}
}

\renewcommand{\arraystretch}{1}


\end{document}